% The real tex generator script
% Creates a complete journal from all your created logs

\documentclass{tufte-book}
    \usepackage{lipsum}             % Used for placeholder text
    
    \newcommand*{\MinYear}{2018}    % Modify these two as you time progresses
    \newcommand*{\MaxYear}{2022}   

%%%%%%%%%%%%%%%%%%%%%%%%%%%%%%%%%%%%%%%%%%%%%%%%%%%%%%%%%%%%%%%%%%%%%%%%%%%%%%%%%%%%%%%%
    \title{My Amazing Journal}      % Title
    \author{Your Amazing Name}      % Author Name
%%%%%%%%%%%%%%%%%%%%%%%%%%%%%%%%%%%%%%%%%%%%%%%%%%%%%%%%%%%%%%%%%%%%%%%%%%%%%%%%%%%%%%%%

    \usepackage[USenglish]{babel}   % For Regional Text.
    \usepackage[useregional]{datetime2}
    \usepackage{pgffor}

    \begin{document}
    
        \maketitle
        \selectlanguage{USenglish}

        % Goes to files in every year/month/ and converts them from markdown to tex
        \foreach \year in {\MinYear,...,\MaxYear}{
            \foreach \month in {1,2,...,12}{
                \foreach \day in {1,2,...,31}{
                    \IfFileExists{../\year/\month/\day.md} {
                        % Displays a section header for the day
                        \section{\DTMdisplaydate{\year}{\month}{\day}{1}}
                        
                        % Goes to files in every year/month/ and converts them from markdown to tex
                        \immediate\write18{pandoc ../\year/\month/\day.md -t latex -o ../texdata/\year/\month/\day.tex}
                        
                        % Concatenates the newly created tex file to the main journal
                        \input{../texdata/\year/\month/\day.tex}
                    } {  
                        % Do Nothing if file does not exist
                    }
                }
            }
        }

    \end{document}